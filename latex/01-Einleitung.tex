%Literatur einfügen     \cite[S.3]{bibkey} 
% In JabRef die zu zitierende Quelle auswählen und STRG+L (z.B. bei Verwendun von TeXstudio) oder STRG+K (hier wird direkt "\cite{bibkey}" kopiert) drücken. Dann mit STRG+V an der Stelle einfügen, an welcher man den Literaturverweis einfügen möchte.
%\section*{Spervermerk} Nicht sichtbares Kapitel in der Gliederung

% In TeXstudio
% Strg+T  Kommentiert Abschnitte oder Zeilen aus 
% Strg+U  Entfernt die auskommentierung


\chapter{Einleitung}
\label{sec:Einleitung} %Textmarke/Positionsmarke, um mit Autoref darauf zu verweisen.


\todotext{Hier soll die eine Einleitung hin..}

Zunächst kommt ein einleitender Text...

\section{Problemstellung / Forschungsfragen}

\section{Lösungsansatz}
\label{sec:loesungsansatz}

\section{Aufbau der Arbeit}
In \autoref{sec:Kapitel2} wird das und das Thema\todo{Bla bla}{}  behandelt...
Dieses und jenes Thema wird in \autoref{sec:Kapitel3} näher betrachtet...

\blindtext



 