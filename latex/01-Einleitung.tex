%Literatur einfügen     \cite[S.3]{bibkey}
% In JabRef die zu zitierende Quelle auswählen und STRG+L (z.B. bei Verwendun von TeXstudio) oder STRG+K (hier wird direkt "\cite{bibkey}" kopiert) drücken. Dann mit STRG+V an der Stelle einfügen, an welcher man den Literaturverweis einfügen möchte.
%\section*{Spervermerk} Nicht sichtbares Kapitel in der Gliederung

% In TeXstudio
% Strg+T  Kommentiert Abschnitte oder Zeilen aus
% Strg+U  Entfernt die auskommentierung


\chapter{Einleitung}
\label{sec:Einleitung}

\section{Hinführung zum Thema und Motivation}
% TODO: Hinführung zum Thema Virtual Tabletop Plugins für Obsidian
% Motivation für Performance-Optimierung in Electron-basierten Anwendungen

\section{Darstellung der Problemstellung}
% TODO: Performance-Probleme bei VTT-Plugins
% Skalierungsprobleme bei großen Datenmengen
% Einschränkungen durch Electron-Architektur

\section{Zielsetzung und Forschungsfrage}
% TODO: Hauptziel: Performance-Optimierung von VTT-Plugins
% Forschungsfrage: Wie können VTT-Plugins in Obsidian performant implementiert werden?
% Teilziele und erwartete Ergebnisse

\section{Aufbau der Arbeit}

Die vorliegende Arbeit gliedert sich in fünf Hauptkapitel:

In Kapitel \autoref{sec:GrundlagenKapitel} werden die theoretischen Grundlagen erarbeitet. Dies umfasst die technischen Rahmenbedingungen von Obsidian und Electron, die konzeptuellen Grundlagen von Virtual Tabletop Tools und Plugin-Architekturen sowie Methoden zur Performance-Analyse.

Kapitel \autoref{sec:KonzeptionImplementierung} beschreibt die Konzeption und Implementierung des VTT-Plugins. Hier werden die Anforderungsanalyse, das Systemdesign, verschiedene Lösungsansätze und die Implementierung der Testumgebung dokumentiert.

In Kapitel \autoref{sec:EvaluationErgebnisse} erfolgt die Evaluation der implementierten Lösung. Die durchgeführten Performance-Messungen werden ausgewertet und verschiedene Optimierungsstrategien verglichen.

Kapitel \autoref{sec:FazitAusblick} fasst die wesentlichen Erkenntnisse zusammen, beantwortet die Forschungsfrage und gibt einen Ausblick auf zukünftige Entwicklungsmöglichkeiten.