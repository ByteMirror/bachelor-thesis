\chapter{Theoretische Grundlagen}
\label{sec:GrundlagenKapitel}

\section{Technische Rahmenbedingungen}

\subsection{Obsidian als Markdown-Editor}
% TODO: Einführung in Obsidian
% Architektur und Plugin-System
% API und Erweiterungsmöglichkeiten

\subsection{Electron Framework}
% TODO: Grundlagen von Electron
% Architektur: Main Process vs Renderer Process
% Performance-Charakteristika
% Speicherverwaltung

\section{Konzeptuelle Grundlagen}

\subsection{Virtual Tabletop Tools (VTTs)}
% TODO: Definition und Zweck von VTTs
% Typische Funktionen (Karten, Token, Würfel)
% Anforderungen an Echtzeit-Interaktivität

\subsection{Plugin-Architekturen}
% TODO: Design Patterns für Plugin-Systeme
% Event-basierte Architekturen
% Datenmanagement in Plugins

\section{Performance-Analyse und Benchmarking}

\subsection{Performance-Metriken}
% TODO: Relevante Metriken (FPS, Latenz, Speicherverbrauch)
% Messverfahren und Tools
% Benchmarking-Standards

\subsection{Optimierungsstrategien}
% TODO: Code-Optimierung
% Algorithmus-Optimierung
% Rendering-Optimierung
% Lazy Loading und Virtualisierung

\section{Stand der Forschung und verwandte Arbeiten}
% TODO: Übersicht über existierende VTT-Lösungen
% Performance-Studien zu Electron-Anwendungen
% Relevante akademische Arbeiten
% Marktübersicht und Vergleich