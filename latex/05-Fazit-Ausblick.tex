\chapter{Fazit und Ausblick}
\label{sec:FazitAusblick}

\section{Zusammenfassung der wesentlichen Erkenntnisse}
% TODO: Kernaussagen der Arbeit
% Wichtigste Ergebnisse der Performance-Analyse
% Erfolgreiche Optimierungsstrategien
% Erreichte Performance-Verbesserungen

\section{Beantwortung der Forschungsfrage}
% TODO: Rückbezug auf die eingangs gestellte Forschungsfrage
% Wie können VTT-Plugins in Obsidian performant implementiert werden?
% Konkrete Antworten basierend auf den Ergebnissen
% Validierung der Hypothesen

\section{Limitationen und kritische Reflexion}

\subsection{Methodische Einschränkungen}
% TODO: Grenzen der Testmethodik
% Nicht untersuchte Aspekte
% Generalisierbarkeit der Ergebnisse

\subsection{Technische Limitationen}
% TODO: Einschränkungen durch Obsidian/Electron
% Hardware-Abhängigkeiten
% Browser-spezifische Probleme

\subsection{Kritische Würdigung}
% TODO: Stärken und Schwächen der Arbeit
% Alternative Ansätze
% Lessons Learned

\section{Ausblick auf zukünftige Entwicklungen}

\subsection{Weiterführende Forschung}
% TODO: Offene Forschungsfragen
% Potenzielle Verbesserungen
% Neue Technologien (WebGPU, WASM)

\subsection{Praktische Anwendung}
% TODO: Integration in bestehende VTT-Systeme
% Übertragbarkeit auf andere Plugin-Typen
% Community-Entwicklung

\subsection{Technologische Trends}
% TODO: Entwicklung von Electron
% Neue Web-Standards
% Alternative Plattformen