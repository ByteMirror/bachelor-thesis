\chapter{Evaluation und Ergebnisse}
\label{sec:EvaluationErgebnisse}

\section{Durchführung der Performance-Messungen}

\subsection{Testumgebung und -bedingungen}
% TODO: Hardware-Spezifikationen
% Software-Versionen
% Netzwerkbedingungen
% Testdaten und -szenarien

\subsection{Messmethodik}
% TODO: Durchführung der Tests
% Anzahl der Testläufe
% Statistische Methoden
% Fehlerbehandlung

\section{Auswertung und Interpretation der Daten}

\subsection{Performance-Metriken}
% TODO: Ladezeiten
% Frame-Rates
% Speicherverbrauch
% CPU-Auslastung

\subsection{Skalierungsverhalten}
% TODO: Tests mit verschiedenen Datenmengen
% Anzahl gleichzeitiger Objekte
% Netzwerk-Performance

\section{Vergleich verschiedener Optimierungsstrategien}

\subsection{Baseline-Performance}
% TODO: Messung ohne Optimierungen
% Identifikation von Bottlenecks

\subsection{Optimierungstechniken}
% TODO: Object Pooling
% Lazy Rendering
% Caching-Strategien
% Event-Batching

\subsection{Vergleichende Analyse}
% TODO: Gegenüberstellung der Ansätze
% Performance-Gewinn pro Technik
% Trade-offs und Kompromisse

\section{Diskussion der Ergebnisse}

\subsection{Interpretation der Messergebnisse}
% TODO: Analyse der Performance-Daten
% Erklärung von Anomalien
% Vergleich mit Erwartungen

\subsection{Best Practices}
% TODO: Empfohlene Optimierungen
% Architektur-Empfehlungen
% Code-Patterns

\subsection{Limitationen}
% TODO: Grenzen der Optimierung
% Technische Einschränkungen
% Trade-offs zwischen Features und Performance